%!TEX TS-program = xelatex
%!TEX encoding = UTF-8 Unicode
%!TEX root = 2020-GS-ARTICLE.tex
%-------------------------------------------------------------------------------
%-------------------------------------------------------------------------------
%	PACKAGES AND OTHER DOCUMENT CONFIGURATIONS
%-------------------------------------------------------------------------------
%-------------------------------------------------------------------------------
\documentclass[
	a4paper,
	twocolumn
	]{article}
%--------------------------------------------------------------- GENERAL SETUP -
\usepackage[T1]{fontenc}
\usepackage[italian]{babel}
\usepackage{
  graphicx,
  dblfloatfix
}
\usepackage{graphicx}
\usepackage{epstopdf}
\epstopdfsetup{update}
\usepackage[usenames]{color}
\usepackage{amssymb}
\usepackage{hyperref} % For hyperlinks in the PDF
%---------------------------------------------------------------- STYLE GS2020 -
%-------------------------------------------------------------------------------
%-------------------------------------------------------------------------------
%	CUSTOM PACKAGES AND OTHER DOCUMENT CONFIGURATIONS
%-------------------------------------------------------------------------------
%-------------------------------------------------------------------------------
\usepackage{Alegreya}
\linespread{1.05}
\usepackage{
	fontspec,
	xltxtra,
	xunicode
	}

\usepackage{
	xfrac,
	unicode-math
	}

\defaultfontfeatures{Mapping=tex-text}
\setmonofont[
	Scale=MatchLowercase
	]{Andale Mono}
\setmathfont[
	Scale=MatchLowercase,
	Scale=1
	]{Libertinus Math}

\usepackage{microtype}
\usepackage[
	top=20mm,
	bottom=25mm,
	textwidth=17.2cm,
	columnsep=0.8cm
	]{geometry}
\usepackage[
	hang,
	small,
	labelfont=bf,
	up,
	textfont=it,
	up
	]{caption}
\usepackage{paralist} % For compact item lists
\usepackage{etoolbox} % Some tools: used for quote environment
\AtBeginEnvironment{quote}{\small}
\usepackage{titling} % Customizing the title section
\usepackage{booktabs} % Horizontal rules in tables
\usepackage{enumitem} % Customized lists
\setlist[itemize]{noitemsep} % Make itemize lists more compact
\usepackage{abstract} % Allows abstract customization
\renewcommand{\abstractnamefont}{\normalfont\bfseries} % Set the "Abstract" text to bold
\renewcommand{\abstracttextfont}{\normalfont\small\itshape} % Set the abstract itself to small italic text
\usepackage{titlesec} % Allows customization of titles
\renewcommand\thesection{\Roman{section}} % Roman numerals for the sections
\renewcommand\thesubsection{\Roman{subsection}} % roman numerals for subsections
\titleformat{\section}[block]{\large\centering}{\thesection.}{1em}{} % Change the look of the section titles
\titleformat{\subsection}[block]{\large}{\thesubsection.}{1em}{} % Change the look of the section titles
%-------------------------------------------------------------------------------
%-------------------------------------------------------------------------------
%	TITLE SECTION
%-------------------------------------------------------------------------------
%-------------------------------------------------------------------------------
\setlength{\droptitle}{-4\baselineskip} % Move the title up
\pretitle{\begin{center}\huge\bfseries} % Article title formatting
\posttitle{\end{center}} % Article title closing formatting
\title{General Relativity \\ \large{\emph{For a more accessible and less technical introduction to this topic, \\ see Introduction to General Relativity}}} % Article title
\author{%
\textsc{Orgo Wikipedio}\\%[1ex]%
\normalsize Conservatorio S. Cecilia di Roma \\ % Your institution
\normalsize orgo @ wikipedio.com % Your email address
\and % Uncomment if 2 authors are required, duplicate these 4 lines if more
\textsc{Wikio Orgopedio}\\%[1ex]%
\normalsize Conservatorio S. Cecilia di Roma \\ % Your institution
\normalsize wikio @ orgopedio.com % Your email address
}
\date{} % Leave empty to omit a date

\usepackage{fancyhdr} % Headers and footers
\pagestyle{fancy} % All pages have headers and footers
\fancyhead{} % Blank out the default header
\fancyfoot{} % Blank out the default footer
\fancyhead[C]{\small Wikipedia • General Relativity} % Custom header text
\fancyfoot[RO,LE]{\small \today~ • w: \input{includes/words.txt} • c: \input{includes/char.txt} • p:~\thepage} % Custom footer text
%-------------------------------------------------------------------------------
%-------------------------------------------------------------------------------
%	LISTINGS
%-------------------------------------------------------------------------------
%-------------------------------------------------------------------------------
\usepackage{listings}
% lstlistings setup
\definecolor{yobg}{rgb}{0.9,0.9,1}
\definecolor{yotxt}{rgb}{0.01,0.01,0.52} % a dark blue.
\definecolor{mylstbg}{rgb}{0.98,0.98,0.98} % a really pale grey.
\definecolor{mylstcmt}{rgb}{0.01,0.52,0.01} % a dark green.
\definecolor{mylstdoc}{rgb}{0.80,0.30,0.80} % a medium pink.

\lstset{%
  aboveskip=10pt,
	belowskip=5pt,
  language=C++,
  numbers=none,%left,%none,
  tabsize=4,
  %frame=single,
  breaklines=true,
  numberstyle=\tiny\ttfamily,
  backgroundcolor=\color{mylstbg},
  basicstyle=\footnotesize\ttfamily,
  commentstyle=\slshape\color{mylstcmt}, %\itshape,
  %frameround=tttt,
  columns=flexible, %fixed,
  showstringspaces=false,
  emptylines=2,
  inputencoding=utf8,
  extendedchars=true,
  literate=	{á}{{\'a}}1
			{à}{{\`a}}1
			{ä}{{\"a}}1
			{â}{{\^a}}1
			{é}{{\'e}}1
			{è}{{\`e}}1
			{ë}{{\"e}}1
			{ê}{{\^e}}1
			{ï}{{\"i}}1
			{î}{{\^i}}1
			{ö}{{\"o}}1
			{ô}{{\^o}}1
			{è}{{\`e}}1
			{ù}{{\`u}}1
			{û}{{\^u}}1
			{ç}{{\c{c}}}1
			{Ç}{{\c{C}}}1,
  emph={component, declare, environment, import, library, process},
  emph={[2]ffunction, fconstant, fvariable},
  emph={[3]button, checkbox, vslider, hslider, nentry, vgroup, hgroup, tgroup, vbargraph, hbargraph, attach},
  emphstyle=\color{yotxt}, %\underline, %\bfseries,
  morecomment=[s][\color{mylstdoc}]{<mdoc>}{</mdoc>},
  rulecolor=\color{black}
}

%-------------------------------------------------------------------- ABSTRACT -
\renewcommand{\maketitlehookd}{%
\begin{abstract}
\noindent\input{includes/abstract.txt}
\end{abstract}
}
%-------------------------------------------------------------------------------
%-------------------------------------------------------------------------------
%	BEGIN DOCUMENT
%-------------------------------------------------------------------------------
%-------------------------------------------------------------------------------
\begin{document}
\maketitle
\thispagestyle{empty}
%-------------------------------------------------------------------------------
%-------------------------------------------------------------------------------
\section*{INTRODUZIONE}
% puntatore a blumlein \cite{ab58}
% puntatore a curtisroads \cite{cr96cmt}
% puntatore a faust \cite{faust}
% puntatore a produzione musicale \cite{pdm}
Inizialmente si è resa necessaria la ricerca di un linguaggio informatico per cui fosse possibile uno studio ed un’assimilazione che si concretizzasse nella finalità di un progetto comprensibile a noi studenti, nonché duttile nella lavorazione ed adattabile al percorso di studi musicale che stiamo intraprendendo.
 In un primo momento, la semplicità nella concezione di un sintetizzatore digitale ci ha fatto comprendere come in realtà uno strumento di questo tipo si apre ad infinite possibilità di sviluppo, scelte personali volte a ricercare risultati adatti alle nostre esigenze.
Tutte queste possibilità, accrescono consciamente le abilità nel gestire i linguaggi base, nonostante quindi gli stretti tempi di analisi per argomenti e impieghi complessi, come detto dalle infinite vie percorribili.
 Il progetto si è indirizzato verso la realizzazione di uno strumento complesso e personalizzabile, per la composizione digitale, un sintetizzatore digitale. A piccoli passi abbiamo sviluppato elaborando ed aggiungendo i vari moduli che vanno ad arricchire l’esperienza interattiva, siamo partiti da preset già concepiti per modellarne di nostri ed appunto integrarli nell’insieme. Senza conoscenze pregresse, abbiamo avuto come obbiettivo lo sviluppo del nostro “Superstereo Synth”.
%-------------------------------------------------------------------------------
\section*{SVILUPPO DEL CODICE}
La prima sezione Graphics User Interfaces (GUI) rende possibile la visualizzazione in moduli dei tools ordinati ed accessibili, nella parte centrale il controllo Master per l’intensità sonora e il ritorno visivo di verifica dell’uscita audio, nella parte sinistra abbiamo il macro modulo di sintesi sottrattiva “SUBTRACTIVE”  che ci permette di applicare un filtro alla una sorgente sonora da un punto di vista spettrale "sottraendo" o modulando da essa bande di frequenze o singole parziali, infine nella parte destra abbiamo il macro modulo denominato “Phaser” che ci permette attraverso la duplicazione del segnale originale, ritardando di qualche millisecondo la copia, di variare la fase in uscita sommando le fasi ottenute dalle due copie.

\begin{figure}[h]
\begin{center}
\includegraphics[width=.47\textwidth]{img/GraphDes}
\caption{\textbf{GUI}. Interfaccia grafica del synth.}
\label{gr01}
\end{center}
\end{figure}

Di seguito la struttura primaria del programma, sia in forma di codice che
rappresentata attraverso un diagramma a blocchi

\begin{lstlisting}
process = subtractive : phaser : master : mutes : meters;
\end{lstlisting}

\begin{figure}[h]
\begin{center}
\includegraphics[width=.47\textwidth]{img/process}
\caption{\textbf{Process}. Diagramma a blocchi della struttura del synth.}
\label{gr01}
\end{center}
\end{figure}

\subsubsection*{GUI}
 Per facilitare la comprensione, abbiamo delineato in prima istanza l’interfaccia grafica, nominando i moduli e assegnandovi gli spazi noti. Lo studio sull’utilizzo dell’interfaccia utente ci ha permesso di ottimizzare gli spazi, rendendo simmetrico ed ordinato il tutto, il codice amplia le possibilità di personalizzare valori numerici, letterali e degli spazi a schermo.
%-----------------------------------------------------
%-------------------------larghezza massima del codice
\begin{lstlisting}
// GUI
main(x) = hgroup("[01] Superstereo Synth",x);
s_g(x) = main(vgroup("[01] SUBTRACTIVE",x));
m_g(x) = main(vgroup("[03] MASTER",x));
p_g(x) = main(vgroup("[05] PHASER",x));
lmeter(x) = main(attach(
	x,an.amp_follower(0.150,x) : ba.linear2db :
	vbargraph("[02] L [unit:dB]", -70,0)));
rmeter(x) = main(attach(
	x,an.amp_follower(0.150,x) : ba.linear2db :
	vbargraph("[04] R [unit:dB]", -70,0)));
meters = lmeter, rmeter;
\end{lstlisting}

\caption{\textbf{GUI}. Il codice che riguarda l'interfaccia utente.}

\subsubsection*{MASTER}

\begin{figure}[h]
\begin{center}
\includegraphics[width=.47\textwidth]{img/Master}
\caption{\textbf{MASTER}. Diagramma a blocchi del modulo Master.}
\label{gr01}
\end{center}
\end{figure}

Il modulo “Master” posto nella parte centrale, è adibito al controllo in entrata di tutto il flusso e le sue variazioni applicate in fase di costruzione del segnale.
Come si può vedere dal diagramma a blocchi, l’imposizione del pulsante “Mute” che interrompe il segnale, chiudendo il canale di uscita era necessario, tanto quanto il fader verticale per il controllo del volume, pratico ed al centro del controller, ci permette di agire direttamente sui valori in scala logaritmica più consona a questo utilizzo.
In ultimo ai lati troviamo i due Meters che ci permettono di verificare visivamente  l’uscita audio dal canale di sinistra e di destra (Left Meters & Right Meters).


%-----------------------------------------------------
%-------------------------larghezza massima del codice
\begin{lstlisting}
// MASTER CONTROLS
mute = m_g(*(1-(checkbox("[04] Mute")))) : si.smoo;
volume = m_g(vslider("[02] VOLUME ",-6,-70,12,0.1)) :
   ba.db2linear : si.smoo;
bpc = p_g(checkbox("[01] Bypass"));
phaser = ba.bypass1to2(bpc,phchop);
mutes = mute,mute;
master = (*(volume), *(volume));
\end{lstlisting}

\caption{\textbf{MASTER}. Il codice interessato.}

\subsubsection*{SUBTRACTIVE}

\begin{figure}[h]
\begin{center}
\includegraphics[width=.47\textwidth]{img/Subtractive}
\caption{\textbf{SUBTRACTIVE}. Diagramma a blocchi primario del modulo Subtractive.}
\label{gr01}
\end{center}
\end{figure}

Come sopracitato nell’introduzione sullo sviluppo del codice, il modulo "Subtractive" riguarda la sintesi sottrattiva del segnale sonoro, abbiamo arricchito questo modulo con vari applicativi, un pulsante di switch tra white noise e un’onda non sinusoidale a dente di sega, questo ci permette di creare due segnali completamente diversi e variare tra un segnale che può essere costante o che abbia cambiamenti periodici nella sua forma d’onda.
Dopo la generazione primaria il segnale entra nel filtro dinamico “moog ladder” dove è possibile variare il controllo f-cut  per filtrare le frequenze e il Q factor, il tutto è aiutato da un controllo di ampiezza per il gain sonoro.

%-----------------------------------------------------
%-------------------------larghezza massima del codice
\begin{lstlisting}
// SUBTRACTIVE
// Sawfreq Custom
sawfreq = 123;
generator = (no.noise *switch),
	(os.sawtooth(sawfreq)*(1-switch)) :> _;
switch = s_g(checkbox("[01] Saw/Noise")) : si.smoo;
gain = s_g(
	vslider("[04] Gain [style:knob]",-12,-96,+12,0.01))
	: ba.db2linear : si.smoo;
Q = s_g(
	vslider("[03] Q [style:knob]",5,0.7072,25,0.01));
fcut = s_g(
	vslider("[02] Cut [style:knob]",0.65,0,1,0.001)) :
	si.smoo;
subtractive = generator : ve.moogLadder(fcut,Q) :
	*(gain);
\end{lstlisting}

\begin{figure}[h]
\begin{center}
\includegraphics[width=.47\textwidth]{img/Moog Ladder}
\caption{\textbf{MOOG LADDER FILTER}. L'originale prevetto risalente al 1966 depositato da Robert Moog.}
\label{gr01}
\end{center}
\end{figure}

\subsubsection*{PHASER}

In ultimo il “PHASER” che si posiziona nella parte sinistra dell’interfaccia, come un phaser tradizionale è composto da una serie di variabili N-allpass che alterano le fasi delle diverse componenti di frequenza del segnale.
Il modulo accoglie come ci mostra il diagramma a blocchi, il “Superstereophaser” per la gestione dei 3 controlli principali:
Low frequency oscillator “LFO” che genera forme d’onda a frequenza infrasonora, modulando gli effetti applicati e creando ad esempio un tremolo.
Feedback, in uscita reintroducendo il segnale all'ingresso possiamo ottenere un effetto più intenso, creando risonanza enfatizzando le frequenze.
Delay per creare ulteriori variazioni da applicare al nostro ciclo sonoro.



\begin{figure}[h]
\begin{center}
\includegraphics[width=.47\textwidth]{img/Feedback}
\caption{\textbf{FEEDBACK}. Come agisce il feedback all'interno del ciclo phaser.}
\label{gr01}
\end{center}
\end{figure}

Uno shortcut utile posizionato sopra ai tre comandi principali è il bypass, che ci permette appunto di escludere i nostri controlli del phaser. Il segnale, modulato in uscita verrà accolto dall’hard limiter restringendolo secondo quanto voluto, come i filtri all pass che renderanno il segnale più netto o morbido.

%-----------------------------------------------------
%-------------------------larghezza massima del codice
\begin{lstlisting}
// PHASER
lff = p_g(
	vslider("[02] Lfo [style:knob]",0.358, 0, 16, 0.001))
	: si.smoo;
fbk = p_g(
	vslider("[03] Feedback [style:knob]",
	        -0.689, -0.999, 0.999, 0.001) : si.smoo);
del = p_g(
	nentry("[04] Delay [style:knob]",1, 1, 100, 1));

lfo = os.osc(lff);

phaserLR(N,x,d,g,fb) = x <: l,r
with{
  allpassL(d,g) = (+ <: de.fdelay(
		(ma.SR/2),d),*(-g)) ~ *(g) : mem,_ : +;
  allpassR(d,g) = (+ <: de.fdelay(
		(ma.SR/2),d),*(g)) ~ *(-g) : mem,_ : +;
  apseqL(N,d,g) = seq(i,N,allpassL(d,g));
  apseqR(N,d,g) = seq(i,N,allpassR(d,g));

  l= _, (+:apseqL(N,d,g))~*(fb):> _;
  r=  _, (+:apseqR(N,d,g))~*(-fb):> _;
};

// STEREO SHUFFLER
pot = m_g(
	vslider("[02] WIDE [style:knob]",100,0,200,0.1)) :
	/(100) : si.smoo;
somma = + : /(2);
diff = - : /(2);
sdm = somma,diff;
wide = _, * (sqrt(pot));
stereoshuffle= _,_ <: sdm : wide <: sdm;

superstereophaser = phaserLR(4,_,del,lfo,fbk) :
	stereoshuffle;

// HARD LIMITER
chopper(a) = min(a) : max(-a);

hardlimiter = chopper(0.7) :
	fi.lowpass(12,15000): chopper(0.9) :
	fi.lowpass6e(20000);
phchop = superstereophaser :
	hardlimiter, hardlimiter;
 \end{lstlisting}

\section*{CONCLUSIONE}

Sarò sincero, in un primo momento, data la situazione poco piacevole degli ultimi mesi e le conseguenti difficoltà intercorse nel normale svolgimento delle lezioni, ero perplesso sul buon esito di questo come altri corsi.  In particolare modo,  questo corso pretende molto e tocca argomenti il cui approfondimento può essere infinito, d’altro canto, questa mole di possibilità ed informazioni di certo, per chi ha il piacere di interessarsene può regalare piacevoli soddisfazioni,  sia per applicazioni basiche sia via via più ricche ed elaborate.
Avendo rapporti pregressi con alcuni linguaggi informatici, posso affermare quanto il percorso scelto è stato si impervio, ma guidato e correttamente calcolato per riuscire nel suo intento, creare da zero un applicativo utile e funzionale per le nostre competenze informatiche e musicali.
Il corso ci ha permesso di apprendere costruendo piano piano, seguendo così un apprendimento passo passo che aiuta a destreggiarsi ed a porre le basi per ulteriori sviluppi futuri. Il Synthesizer ha fatto riaffiorare conoscenze e sistemi di funzionamento non solo digitale ma anche analogico altrimenti dimenticati, ci ha permesso di metterci alla prova e godere poi di un applicativo sempre utile e personalizzabile.

\subsubsection*{BIBLIOGRAFIA E SITOGRAFIA}

Blumlein, Alan Dower - British Patent Specification 394,325 (1958), JAES, vol. 6, no. 2, p.91.
Curtis ROADS - The computer music tutorial (1996), The MIT press.
Faust - Faust documentation, https://faustdoc.grame.fr
Portale Produzione Musicale - https://produzionemusicale.com


%-------------------------------------------------------------------------------
% \subsubsection*{UNNUMBERED SUB-SUB-SECTION}
% Some predictions of general relativity differ significantly from those of
% classical physics, especially concerning the passage of time, the geometry of
% space, the motion of bodies in free fall, and the propagation of light. Examples
% of such differences include gravitational time dilation, gravitational lensing,
% the gravitational redshift of light, and the gravitational time delay. The
% predictions of general relativity in relation to classical physics have been
% confirmed in all observations and experiments to date. Although general
% relativity is not the only relativistic theory of gravity, it is the simplest
% theory that is consistent with experimental data. However, unanswered questions
% remain, the most fundamental being how general relativity can be reconciled with
% the laws of quantum physics to produce a complete and self-consistent theory of
% quantum gravity.
%
% Einstein's theory has important astrophysical implications. For example, it
% implies the existence of black holes regions of space in which space and time
% are distorted in such a way that nothing, not even light, can escape as an
% end state for massive stars. There is ample evidence that the intense radiation
% emitted by certain kinds of astronomical objects is due to black holes. For
% example, microquasars and active galactic nuclei result from the presence of
% stellar black holes and supermassive black holes\ldots
%
%
% \vfill\null
%
% \begin{figure}[b]
% \begin{center}
% \includegraphics[width=.47\textwidth]{img/image1.png}
% \caption{\textbf{Spacetime curvature schematic}. Lattice analogy of the deformation
% of spacetime caused by a planetary mass.}
% \label{gr01}
% \end{center}
% \end{figure}
%
% \newpage % USE NEWPAGE TO FORCE COLUMNN INTERRUPTION
% %-------------------------------------------------------------------------------
% %-------------------------------------------------------------------------------
% \section*{UNNUMBERED SECTION}
%
% \begin{quote}
% La musica non e` solo composizione. \\
% Non è artigianato, non è un mestiere. \\
% La musica è pensiero.\cite{nono85}.
% \end{quote}
%
% Some predictions of general relativity differ significantly from those of
% classical physics, especially concerning the passage of time, the geometry of
% space, the motion of bodies in free fall, and the propagation of light. Examples
% of such differences include gravitational time dilation, gravitational lensing,
% the gravitational redshift of light, and the gravitational time delay. The
% predictions of general relativity in relation to classical physics have been
% confirmed in all observations and experiments to date. Although general
% relativity is not the only relativistic theory of gravity, it is the simplest
% theory that is consistent with experimental data. However, unanswered questions
% remain, the most fundamental being how general relativity can be reconciled with
% the laws of quantum physics to produce a complete and self-consistent theory of
% quantum gravity.
%
% \begin{table}[htp]
% \begin{center}
% \begin{tabular}{ll}
% \textbf{Stages} & \textbf{Dur.} \\
% \hline
% \textbf{Omnidirectional Expositions} & 6 mo. \\
% Sound-shape analysis and visualizations & \\
% Sound-shape reproduction & \\
% Sound-shape database design & \\
% \hline
% \textbf{Micro-Rhythm of sound-shape} & 12 mo. \\
% Solo repertoire analysis & \\
% Sound-shape explosion in practising & \\
% From literature to shapes open-data & \\
% \hline
% \textbf{Rhythm of sound-shape interactions} & 12 mo. \\
% Multiple sources multiple shapes & \\
% Relationship and complexity perception & \\
% \hline
% \textbf{Sound-shape in musical composition} & 12 mo. \\
% AI: unleashed writing opportunities & \\
% AI: can you listen the time? & \\
% \hline
% \textbf{Final documentation} & 6 mo. \\
% \end{tabular}
% \label{timesheet}
% \caption{Thinking Tetrahedral Today stages}
% \end{center}
% \end{table}%
%
% Einstein's theory has important astrophysical implications. For example, it
% implies the existence of black holes regions of space in which space and time
% are distorted in such a way that nothing, not even light, can escape as an
% end state for massive stars. There is ample evidence that the intense radiation
% emitted by certain kinds of astronomical objects is due to black holes. For
% example, microquasars and active galactic nuclei result from the presence of
% stellar black holes and supermassive black holes, respectively. The bending of
% light by gravity can lead to the phenomenon of gravitational lensing, in which
% multiple images of the same distant astronomical object are visible in the sky.
% General relativity also predicts the existence of gravitational waves, which
% have since been observed directly by the physics collaboration LIGO. In addition,
% general relativity is the basis of current cosmological models of a consistently
% expanding universe.
%
% \begin{compactitem}
% \item Derivations of the Lorentz transformations
% \item Einstein–Hilbert action
% \item Tests of general relativity
% \item Two-body problem in general relativity
% \end{compactitem}
%
% \begin{figure}[t]
% \centering
% \includegraphics[width=.47\textwidth]{img/image2.jpg}
% \caption{Mind Mapping}
% \label{gs}
% \end{figure}
%
% \begin{equation}
% m(x,p,\theta) = (p*x) + ((1-p)*(x\cos\theta)
% \label{eq:mid}
% \end{equation}
%
% Some predictions of general relativity differ significantly from those of
% classical physics, especially concerning the passage of time, the geometry of
% space, the motion of bodies in free fall, and the propagation of light.
%
% %--------------------------------------------
% %----------------larghezza massima del codice
% \begin{lstlisting}
% mspan(x,p,rad) = m,s
% with{
%   m = (p*x)+((1-p)*(x*cos(rad)));
%   s = x*(sin(-rad));
% };
% \end{lstlisting}
%
% Examples
% of such differences include gravitational time dilation, gravitational lensing,
% the gravitational redshift of light, and the gravitational time delay. The
% predictions of general relativity in relation to classical physics have been
% confirmed in all observations and experiments to date. Although general
% relativity is not the only relativistic theory of gravity, it is the simplest
% theory that is consistent with experimental data.

\vfill\null

\raggedright
\bibliographystyle{plain}
\bibliography{includes/bibliography.bib}

\end{document}

%%%%%%%%%%%%%%%%%%%%%%%%%%%%%%%%%%%%%%%%%%%%%%%%%%%%%%%%%%%%%%%%%%%%%%%%%%%%%%%%
% 2020 GIUSEPPE SILVI ARTICLE TEMPLATE BASED ON
%%%%%%%%%%%%%%%%%%%%%%%%%%%%%%%%%%%%%%%%%%%%%%%%%%%%%%%%%%%%%%%%%%%%%%%%%%%%%%%%
% Journal Article
% LaTeX Template
% Version 1.4 (15/5/16)
% This template has been downloaded from:
% http://www.LaTeXTemplates.com
% Original author:
% Frits Wenneker (http://www.howtotex.com) with extensive modifications by
% Vel (vel@LaTeXTemplates.com)
% License:
% CC BY-NC-SA 3.0 (http://creativecommons.org/licenses/by-nc-sa/3.0/)
%%%%%%%%%%%%%%%%%%%%%%%%%%%%%%%%%%%%%%%%%%%%%%%%%%%%%%%%%%%%%%%%%%%%%%%%%%%%%%%%
